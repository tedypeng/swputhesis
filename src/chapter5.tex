% !TEX root = ../swputhesis.tex
\chapter{总结与展望}
\section{全文总结}
本次研究受到FCOS目标检测器性能的启发,采用设计了一款单阶段全景分割网络,其包括三个分支结构,分别为目标检测、语义和全景分支结构,以此来扩展目标检测器输出的网络特征图。而目标检测器的主要作用在于对实例中心目标进行检测以及感知生成像素级实例掩码。语义分支的主要作用在于给全景分支提供语义标签。计算全景分割结果的原理在于对语义和全景分支网络的像素进行乘积计算。以端对端的学习方式防止网络产生启发式融合冲突问题。本次研究设计的网络架构是自上而下的单阶段结构,推理的难度较小,能够降低深度学习推理的难度,更容易部署实际应用场景。在对比分析数据集的分割可视化效果来看,单阶段架构的分割效果更理想,通过感知像素级实例掩码的方式能够有效提高全景分割的准确性,这样就能够有效的解决全景分割处理二维图像过程中出现的推理准确性和时间不平衡的问题。

\section{工作展望}
为了进一步提高单阶段结构全景分割处理二维图像的准确性,本章节在感知像素级实例掩码的基础上提出了全新的全景分割方法,其主要包括三个分支结构,即目标检测、语义和全景分支结构,以此来增加FCOS目标检测器输出网络特征的结果。而目标检测器的主要作用在于对实例中心目标进行检测以及感知生成像素级实例掩码。语义分支的主要作用在于给全景分支提供语义标签。计算全景分割结果的原理在于对语义和全景分支网络的像素进行乘积计算。本次设计的网络选择的验证和标准数据集是Cityscapes ,那么本次设计的全景分割方法的性能与双阶段全景分割法的性能是非常接近的。在对比验证分割预测和分析可视化程度方面,本次研究在感知像素级实例掩码的基础上提出了全景分割二维图像的方法,能够有效的处理大尺度物体过于分割而小尺度问题分割不足的问题,以此来处理推理时间和进度不平衡的问题,使模型的推理更加容易,便于模型进行深度学习,让布局实际应用场景变得更加容易。
