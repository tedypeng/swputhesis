\begin{cabstract}
    图像分割是一个计算机视觉领域实现对图像分析的基础领域,但伴随数据量的剧增、图像场景越来越复杂,单一的图像语义或实例分割无法满足多数计算机视觉的任务需求,因此全景分割应运而生。全景分割利用语义标签和实例特征实现对图像场景的全面理解。当前高精度全景分割方法计算量大,导致高效方法不能满足实际应用的时效需求。精度和耗时的冲突已成为二维图像和三维激光雷达图像全景分割技术,在部署自动驾驶应用场景中瓶颈。为缓解此问题,本文开展了如下研究:首先采用一种基于像素级实例感知的全景分割方法,用于提高基于单阶段架构的二维图像全景方法的分割精度。该方法利用FPN(Feature Pyramid Network,特征金字塔网络)对输入图像进行多尺度特征信息提取,再将多尺度特征图分别传入语义分支、目标检测分支和全景分支。网络中使用FCOS(Fully Convolutional One-stage,全卷积单阶段)目标检测器为全景分支提供实例中心信息,辅助像素级实例感知掩码生成,语义分支为全景分支提供标签。基于所属方法获得高精度分割效果的同时,使用语义分支与全景分支乘积作为全景分割结果,减少了后处理融合耗时。本研究选择Cityscapes数据集(包含超过5000张像素图像)用以实验。 
    该方法不仅完成了基于特征金字塔和像素级实例感知掩码提高全景分割精度的任务,还实现了使用语义分支与全景分支乘积融合代替传统融合处理减少计算耗时的目的,缓解了全景分割方法精度和效率之间的经典折衷问题。在Cityscapes数据集上实验论证了所属方法的良好分割效果,为自动驾驶等计算机视觉任务的进一步应用,丰富了可用于图像分割技术手段。

    摘要题头应居中,小二号黑体 (与章标题格式相同)。
    摘要要求为写实性的叙述,阐明研究意义、理论方法、开展的工作、取得成果和认识,最好给出定量性的结论。
    英文摘要应与中文内容一致,表述得体。

    摘要文字后空一行,顶格 (即不缩进) 写出关键词。``关键词:'' 用小四号、黑体,
    而关键词内容用小四号、宋体,内容 3 $\sim$ 5 个,关键词之间用  ``;'' 隔开。

    ``Key words'' 没给出格式规定,几个 Word 模板都有不同,有采用宋体,有采用黑体的,
    这里暂时采用了宋体加粗。

    \ckeywords{西南石油大学;硕士论文;模板;深度学习;全景分割;卷积神经网络;特征金字塔网络;FCOS目标检测器}

\end{cabstract}

\begin{eabstract}
    The abstract is an important component of your thesis. Presented at the
    beginning of the thesis, it is likely the first substantive description of
    your work read by an external examiner. You should view it as an opportunity
    to set accurate expectations.

    The abstract is a summary of the whole thesis. It presents all the major
    elements of your work in a highly condensed form.
    An abstract often functions, together with the thesis title,
    as a stand-alone text.

 
    \ekeywords{Deep learning; Panoramic segmentation; Feature Pyramid Network; Convolutional neural network; FCOS target detector}
\end{eabstract}
